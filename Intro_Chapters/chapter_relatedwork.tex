


\chapter{Related Work}
\label{chp:relatedwork}

Faced with the challenge of providing an interactive virtual
environment for authoring plastic surgery simulations, the field of
computer graphics research has generated many potential solutions and
techniques to solve this problem. Procedural techniques
~\citep{JoshiMDGS:2007,WangP:2002,KavanCZO:2008,VaillBGCRWGP:2013} offer real-time
performance for certain animation tasks but lack the physical accuracy
needed in surgical simulations. Consequently, some research ventures
into surgical simulation turned to elastic deformation models
~\citep{TerzoPBF:1987} that responded more realistically to scenarios of
probing and cutting ~\citep{BroC:1996,MendoL:2003,NienhS:2001}. However,
these early works were limited in their scope and effectiveness due to
computational cost, geometric constraints and oversimplified material
models. In this chapter we outline prior contributions that help
address these limitations, and review a number of existing surgical
simulation systems.

\subsection{Simulation of Elastic Materials}

Simulation of elastic deformable models is ubiquitous in computer
graphics and remains a vibrant area of research. Algorithmic
techniques for deformable body simulation, pioneered by \citet{TerzoPBF:1987} have attained a significant level of
maturity, leading to broad adoption in visual effects, games, virtual
environments and biomechanics applications. Over the years, a number
of different approaches have been explored for the simulation of
deformable objects, each with different strengths. 

Lattice-based volumetric deformers are popular components in both
physics-based and procedural animation techniques. In the case of
physics-based simulation, one of their key advantages is that they
avoid having to construct a simulation-ready conforming volume mesh,
which is a delicate preprocessing task often requiring supervision and
fine-tuning. Another crucial benefit is that the regularity of such
data structures enables aggressive performance optimizations as
vividly demonstrated by shape matching techniques
~\citep{RiverJ:2007}. Cartesian lattices have also been leveraged to
accelerate performance in physics-based approaches, albeit
predominantly for simple models such as linear or corotated elasticity
~\citep{MuellTG:2004,GeorgW:2008,McAdaZSETTS:2011}. Prior graphics
work, however, has not demonstrated such aggressive performance gains
from lattice-based discretizations when highly nonlinear, anisotropic
or incompressible materials are involved. In part, this is attributed
to the fact that simulation of complex materials commands an increased
level of attention to issues of robust convergence. Mature solutions
to these concerns have predominantly been demonstrated in the context
of specific discretizations (e.g. explicit tetrahedral meshes) where
regularity of data structures, compactness of memory footprint and
parallelization/vectorization potential were not inherently
emphasized. Furthermore, as applications requiring the use of complex
materials are also likely to emphasize geometric accuracy, they often
opt for conforming mesh discretizations due to their superior
performance in capturing intricate boundary features, even if their
computational cost is higher.

Despite the popularity and feature set of conforming discretizations,
many researchers have explored improving embedded techniques for
reasons of simplicity and performance. Embedding has been combined
with homogenization ~\citep{NesmePF:2006,KhareMOD:2009} to resolve
sub-element variation of material parameters, optionally with the use
of non-manifold embedding lattices to support objects with a branching
structure ~\citep{NesmeKJF:2009}. 
\citet{JerabBBFA:2010} employed a method similar to the one
presented in Chapter \ref{chp:nonmanifold}, using a finer voxel grid
to capture material topology to be embedded in a coarser, non-manifold
voxel grid. Finally, \citet{ZhaoB:2013}
demonstrated the use of multiple voxel grid domains to segment a model
hierarchically, which they used to simulate plants at interactive
rates.

In addition to classical FEM approaches, some authors have
achieved success with more exotic variations. Extended FEM (XFEM)
formulations have also been explored ~\citep{JerabK:2009}, where
discontinuities are introduced into the element's shape functions, to
model cutting. In a similar vein,
\citet{KaufmMBG:2009} used discontinuous Galerkin FEM
formulations. Others have dispensed with mesh based discretizations
completely, preferring meshless methods ~\citep{DeB:2000} which were
also used for surgical simulation ~\citep{DeKLS:2005}.

\subsection{Anatomical Modeling}
Approaches based on the FEM have been particularly popular in the
medical simulation community ~\citep{MarchADC:2008} where the need for
biologically accurate materials is more pronounced. In one of the
earliest uses of advanced materials in computer animation, \citet{ChenZ:1992} focused on anatomical structures such
as muscles. FEM techniques were further leveraged in the animation
literature for the discretization of linear elasticity for fracture
modeling in a small-strain regime ~\citep{OBriH:1999}. Highly
nonlinear materials such as active musculature \citet{TeranBHF:2003}
exposed challenges in robustness and numerical stability of FEM
discretizations. Invertible FEM ~\citep{IrvinTF:2004} improved
simulation robustness in scenarios involving extreme compression,
while modified Newton methods ~\citep{TeranSIF:2005} reduced the cost
of implicit schemes with large time steps. Several of these algorithms
have been incorporated in open-source modeling and simulation packages
~\citep{SinSB:2013}. Solutions have also been proposed for material
behaviors such as incompressibility ~\citep{IrvinSF:2007} and
viscoelasticity ~\citep{GokteBO:2004,WojtaT:2008}, both of which can
be found in typical biomaterials.  Recent results in coupled
Lagrangian-Eulerian simulation of solids have also facilitated the
inclusion of intricate contact and collision handling in biomechanical
modeling tasks ~\citep{SuedaKP:2008,LiSNP:2013,FanLP:2014}.

\subsection{Topology Change}

A number of techniques have targeted topology change during
simulation, due to cutting or fracture. Early work
~\citep{TerzoF:1988b} resorted to breaking connectivity of elements
when stress limits were exceeded. Later methods ~\citep{NienhS:2001}
split tetrahedra near cut boundaries and then used vertex snapping to
more accurately approximate the cut. \citet{SteinHGS:2006} used a
similar approach in the context of surgical simulation, where a
combination of node-snapping and edge cuts on tetrahedra were used to
avoid thin elements, while still remaining close to the user's
specified cuts. Local remeshing was also employed to simulate cracks
in brittle materials ~\citep{OBriH:1999}. An issue with such
subdivision schemes is the possible creation of poorly conditioned
elements, which prompted a number of authors to pursue embedded
simulation schemes ~\citep{MolinBF:2004,TeranSBNLF:2005}. These
techniques use non-conforming meshes with elements which are only
partially covered by material, in lieu of conforming
remeshing. Embedded simulation can provide a great degree of
flexibility in cutting and fracture scenarios ~\citep{SifakDF:2007},
although cutting meshes along arbitrary surfaces requires delicate
book-keeping and careful handling of degeneracies.


\subsection{Surgical Simulation}
While much of the previously discussed work is geared towards general
elastic body simulation in computer graphics, many relevant results
originated in surgery-specific work. 
\citet{PiepeLR:1995} demonstrated a very early surgical simulation
platform for facial procedures, using FEM elastic shells.  Many
surgical simulation projects focus on the mechanical manipulation of
organs and other soft internal objects
~\citep{NienhS:2001,KimCDS:2007}. Even expensive commercial simulators
like the Lap Mentor and GI Mentor primarily focus on pushing and
cutting simulated internal organs
~\citep{SUSAC:2002--2014b,SUSAC:2002--2014}. These types of simulations
are so common that several open source frameworks have been built to
specifically support further development
~\citep{AllarCFBPDDGo:2007,CavusGT:2006}. These provide easy access to
common components like haptic feedback and APIs to connect multiple
simulated components. Certain surgical simulation systems are tailored
to specific skills, including interactive simulations of needle
insertion ~\citep{ChentARCHGSO:2009}.

\subsection{Performance Optimization}
Improving simulation rates is a common challenge for many interactive
modeling tasks, and even more so for accuracy-conscious applications
such as virtual surgery. Attempts to improve performance have either
relied on new data structures, faster solvers, or aggressive use of
parallelization. The Boundary Element Method ~\citep{JamesP:1999} has
been used to achieve interactive deformation rates for objects
manipulated via their surface. Other authors have employed similar
formulations that abstract away interior degrees of freedom to
accelerate collision processing ~\citep{GaoMS:2014}. Grid-based,
embedded elastic models
~\citep{MuellTG:2004,NesmePF:2006,McAdaZSETTS:2011,PatteMS:2012,MitchCS:2015}
have been very popular due to their inherent potential for performance
optimizations, and can also be used with shape-matching approaches
~\citep{RiverJ:2007}. They form the foundation for a class of highly
efficient, multigrid-based numerical solution techniques
~\citep{ZhuSTB:2010,GeorgW:2008,DickGW:2011}.  Regular discretizations
have also been coupled with multigrid solvers to facilitate GPU
accelerations for elastic skinning techniques ~\citep{McAdaST:2010}.
However, in spite of the efficiency of multigrid schemes, adapting
them to the presence of incisions or other intricate topological
features can be a nontrivial proposition.

\citet{HermaRF:2009} analyzed data flow in their
simulations to inform a parallel scheduler for multicore systems. To
avoid write hazards during parallel code execution, 
\citet{KimP:2011} proposed a system of computation phases with
coalesced memory writes, which allowed them to parallelize force
computation. Related efforts by 
\citet{CourtA:2009}, developed a parallel version of the
Gauss-Seidel algorithm that can run on GPUs. Finally, optimized direct
solvers have been shown to be very effective ~\citep{SinSB:2013} and
have employed techniques such as delayed updates to factorization
approaches ~\citep{HechtLSO:2012} for improved efficiency. The
approach outlined in Chapter \ref{chp:macroblocks} is related to these
approaches, as well as the general class of Schur complement methods
~\citep{QuartV:1999}.

\subsection{Level Sets and Collision Handling}

Level set methods were first introduced by \citet{OsherS:1988} for tracking moving interfaces in the
context of Hamilton-Jacobi equations.  Subsequently, \citet{AdalsS:1994} proposed substantial runtime savings
by restricting computations to a thin band of active voxels near the
interface.  \citet{Sethi:1998} proposed fast marching
methods for monotonically advancing fronts as well as for redistancing
the level set using values seeded only on the narrow band.  Besides
fast computation, a number of methods have also been proposed for
efficiently storing level sets including octrees~\citep{LosasGF:2004},
RLE representations~\citep{HoustNBNM:2006,IrvinGLF:2006,ChentM:2011},
the VDB data structure~\citep{Muset:2013} which evolved from Dynamic
Tubular Grids~\citep{NielsM:2006} and the DB+Grid data
structure~\citep{Muset:2011}, and the virtual-memory based SPGrid data
structure~\citep{SetalABS:2014}.

Methods have been proposed for computing implicit representations of
non-manifold surfaces~\citep{BloomF:1995,YuanYW:2012}. Similar ideas
were used for simulating bubbles~\citep{ZhengYP:2006} and multiphase
fluids~\citep{LosasSSF:2006}. The work in Chapter \ref{chp:nonmanifold}
diverges from these approaches as we enhance the expressive capability
of a \emph{single} level set by embedding signed distance values on an
explicit mesh. Our work is related to the practice of embedding
high-resolution geometry in regular meshes, a concept that was first
leveraged by \citet{MulleTG:2004} for deformable
body simulations and fracture.  In addition to hexahedral embeddings,
methods such as the virtual node algorithm~\citep{MolinBF:2004} have
been used to create non-manifold tetrahedral lattices that correspond
to thin topological features in the embedding geometry. Virtual node
concepts are also similar to XFEM methods which were used for crack
modeling~\citep{MoeesDB:1999} and for cutting and fracturing thin
shells~\citep{KaufmMBGG:2009}. This principle has continued to evolve
with many of the topological limitations in prior approaches being
raised by \citet{SifakDF:2007} and has been
successfully used in production tools as well~\citep{HellrSSST:2009}.

Our non-manifold level set approach in Chapter \ref{chp:nonmanifold}
is inspired by these methods, but it needs to be made cognizant of
further topological limitations that the signed distance field imposes
on our representation (see
Section~\ref{sec:non-manifold-level-sets}). Notably, when dealing with
collisions near thin features, all of the aforementioned approaches
employed detection and response techniques based on surface
meshes~\citep{BridsFA:2002} that rely on the availability of good
surface meshes, are computationally expensive, presume collision-free
history or use impulses which makes implicit integration
challenging. To accelerate collision detection and response while
allowing for implicit integration, methods have been proposed using
implicit surface representations~\citep{McAdaZSETTS:2011} which work
even in near-interactive settings, but require enough level set
resolution to avoid any non-manifold features altogether. Recently,
image-based techniques~\citep{FaureBAF:2008,WangFP:2012} have been
proposed which provide an interesting alternative.  Finally, implicit
surfaces have also been recently used in real-time skinning
applications~\citep{VaillBGCRWGP:2013,VaillGBWC:2014}.


%%% Local Variables:
%%% mode: latex
%%% TeX-master: "../document"
%%% End:
