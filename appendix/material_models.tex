
\chapter{Material Models}
\label{apx:materialmodels}

\noindent We provide the necessary formulas for implementing the following four material
models referenced in Chapter \ref{chp:engineering}:
\begin{itemize}
\item Linear Elasticity
\item Corotated Linear Elasticity
\item The St.\ Venant-Kirchhoff model
\item Neohookean Elasticity
\end{itemize}

Among the listed physical quantities, $\Psi_0(\mathbf{F})$,
$\mathbf{P}_0(\mathbf{F})$, $M(\mathbf{F})$ and $\mathbf{Q}(\mathbf{F})$ have
been concretely defined in our paper. We define the stress derivative tensor
$\hat{\mathcal{T}}=\partial\hat{\mathbf{P}}/\partial\mathbf{F}$ indirectly, by listing the corresponding sparse rotated tensor
$\hat{\mathcal{T}}^D$, defined by \citet{TeranSIF:2005}. The relation between the
two tensors is given by the equation

$$
\hat{\mathcal{T}}:\delta\mathbf{F}=
\mathbf{U}
\left[
\hat{\mathcal{T}}^D:
\left(
\mathbf{U}^T\delta\mathbf{FV}
\right)
\right]
\mathbf{V}^T
$$

We also provide the sparse rotated form $\mathcal{T}^D$ of the stress derivative
tensor $\mathcal{T}=\partial\mathbf{P}/\partial\mathbf{F}$ which is obtained if
the conventional approach of \citet{TeranSIF:2005} is followed, instead of our
pressure-augmented mixed formulation. The definiteness fix described in our
paper amounts to a modification of $\hat{\mathcal{T}}^D$ as follows:

$$
\hat{\mathcal{T}}^D_{\mbox{\small fixed}}:=\hat{\mathcal{T}}^D(\mu;\lambda)+\mbox{\textbf{Def}}\left\{\mathcal{T}^D(\mu;\lambda^\ast)\right\}-\mathcal{T}^D(\mu;\lambda^\ast)
$$

In this expression $\mbox{\textbf{Def}}\left\{\cdot\right\}$ indicates the
projection of the sparse rotated stress tensor to its symmetric positive
definite part (as detailed in \citet{TeranSIF:2005}), and $\lambda^\ast$ is an
adjusted Lam\'{e} coefficient corresponding to \emph{moderate} ($\nu<0.4$) incompressibility.

\paragraph{List of symbols}

\begin{itemize}
\item $\bvec{\epsilon}:=\frac{1}{2}\left(\mathbf{F}+\mathbf{F}\right)-\mathbf{I}$ is
    the infinitesimal strain tensor (also known as Cauchy's strain tensor, or
    small strain tensor).
\item $\mathbf{E}:=\frac{1}{2}\left(\mathbf{F}^T\mathbf{F}-\mathbf{I}\right)$ is
  the nonlinear Green strain tensor.
\item Matrices $\mathbf{R}$ and $\mathbf{S}$ are the factors of the Polar
  Decomposition $\mathbf{F}=\mathbf{RS}$.
\item Matrices $\mathbf{U},\ \mathbf{\Sigma}$ and $\mathbf{V}$ are the factors
  of the Singular Value Decomposition $\mathbf{F}=\mathbf{U\Sigma
    V}^T$. The singular values are denoted by $\sigma_i=[\Sigma]_{ii}$.
\item $I_1:=\|\mathbf{F}\|_F^2=\tr(\mathbf{F}^T\mathbf{F})$ is the first
  isotropic invariant.
\item $J:=\det(\mathbf{F})$ is the volume change ratio.
\end{itemize}

\begin{landscape}

%\thispagestyle{empty}
\begin{table}[h]
%\small
\centering
\begin{tabular}{|C{1in}|C{1.75in}|C{1.75in}|C{1.75in}|C{1.75in}|}
\hline
\newline& Linear Elasticity & Corotated Linear Elasticity & St.\ Venant-Kirchhoff  & Neohookean Material \\
\hline
\hline
$\Psi_0(\mathbf{F})$
	& $\mu\|\bvec{\epsilon}\|_F^2$
		&  \newline $\mu\|\mathbf{F}-\mathbf{R}\|_F^2$\mbox{\small, or} \newline$\mu\|\mathbf{S}-\mathbf{I}\|_F^2$\mbox{\small, or} $\mu\|\mathbf{\Sigma}-\mathbf{I}\|_F^2$ \newline
			& $\mu\|\mathbf{E}\|_F^2$
				& $\frac{\mu}{2}(I_1-3)-\mu\log(J)$ \\
\hline
$\mathbf{P}_0(\mathbf{F})$
	& $2\mu\bvec{\epsilon}$
		&  $2\mu(\mathbf{F}-\mathbf{R})$
			& $2\mu\mathbf{FE}$
				& \newline $\mu(\mathbf{F}-\mathbf{F}^{-T})$ \newline \\
\hline
$M(\mathbf{F})$
	& $\tr(\bvec{\epsilon})$
		&   \newline $\tr(\mathbf{S}-\mathbf{I})$\mbox{\small, or} $\tr(\mathbf{\Sigma}-\mathbf{I})$\newline
			& $\tr(\mathbf{E})$
				& $\log(J)$ \\
\hline
$\mathbf{Q}(\mathbf{F})$
	& \newline $\mathbf{I}$  \newline
		&  $\mathbf{R}$
			& $\mathbf{F}$
				& $\mathbf{F}^{-T}$ \\
\hline
$\delta\mathbf{\hat{P}}(\delta\mathbf{F},\delta p;\mathbf{F},p)$
	& \newline $\mu(\delta\mathbf{F}+\delta\mathbf{F}^T)+\alpha\delta p\mathbf{I}-\alpha^2p/\kappa$  \newline
		&  \emph{Use tensor $\mathcal{T}$ instead}
			& \emph{Use tensor $\mathcal{T}$ instead}
				& \emph{Use tensor $\mathcal{T}$ instead} \\
\hline
$\hat{\mathcal{T}}^D$ \newline ($i\ne j$ implied \newline in  formulas)
	& \emph{Not an isotropic model -- Use explicit $\delta\mathbf{\hat{P}}$ formula}
		&  \newline $\hat{T}_{iiii}=2\mu$ \newline $\hat{T}_{iijj}=0$ \newline $\hat{T}_{ijij}=2\mu+\frac{\alpha p-2\mu}{\sigma_i+\sigma_j}$ \newline  $\hat{T}_{ijji}=\frac{2\mu-\alpha p}{\sigma_i+\sigma_j}$ \newline
			& \newline $\hat{T}_{iiii}=\mu(3\sigma_i^2-1)+\alpha p$ \newline $\hat{T}_{iijj}=0$ \newline $\hat{T}_{ijij}=\mu(\sigma_i^2+\sigma_j^2-1)+\alpha p$ \newline  $\hat{T}_{ijji}=\mu\sigma_i\sigma_j$ \newline
				& \newline $\hat{T}_{iiii}=\mu+\frac{\mu-\alpha p}{\sigma_i^2}$ \newline $\hat{T}_{iijj}=0$ \newline $\hat{T}_{ijij}=\mu$ \newline  $\hat{T}_{ijji}=\frac{\mu-\alpha p}{\sigma_i\sigma_j}$ \newline \\
\hline
$\mathcal{T}^D$ \newline ($i\ne j$ implied \newline in  formulas)
	& \emph{Not an isotropic model -- Use explicit $\delta\mathbf{\hat{P}}$ formula}
		&  \newline $T_{iiii}=2\mu+\kappa$ \newline $T_{iijj}=\kappa$ \newline $T_{ijij}=2\mu+\frac{\kappa\str(\mathbf{\Sigma}-\mathbf{I})-2\mu}{\sigma_i+\sigma_j}$ \newline  $T_{ijji}=\frac{2\mu-\kappa\str(\mathbf{\Sigma}-\mathbf{I})}{\sigma_i+\sigma_j}$ \newline
			& \newline $T_{iiii}=\sigma_i^2(3\mu-\kappa)+$\newline$\hspace*{.6in}+\frac{\kappa}{2}\tr(\mathbf{\Sigma}^2-\mathbf{I})-\mu$ \newline $T_{iijj}=\kappa\sigma_i\sigma_j$ \newline $T_{ijij}=\mu(\sigma_i^2+\sigma_j^2-1)+$\newline$\hspace*{.2in}+\frac{\kappa}{2}\tr(\mathbf{\Sigma}^2-\mathbf{I})$ \newline  $T_{ijji}=\mu\sigma_i\sigma_j$ \newline
				& \newline $T_{iiii}=\mu+\frac{\mu-\kappa[\log(J)-1]}{\sigma_i^2}$ \newline $T_{iijj}=\frac{\kappa}{\sigma_i\sigma_j}$ \newline $T_{ijij}=\mu$ \newline  $T_{ijji}=\frac{\mu-\kappa\log(J)}{\sigma_i\sigma_j}$ \newline \\
\hline\end{tabular}
\label{tab:mat2}
\end{table}

\end{landscape}

%%% Local Variables:
%%% mode: latex
%%% TeX-master: "../document"
%%% End:
