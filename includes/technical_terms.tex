\newacronym[%
description={A Practical Visual Interactive System is a dynamic
virtual environment where by user input produces changes in state for
the purpose of supporting some task.}]
{pvis}{PVIS}{practical visual interactive system}

\newacronym[%
description={A Head Mounted Display is a virtual reality display
  device that fits over a user's head, providing them with an isolated
  and personal immersive experience into virtual environments.}]
{hmd}{HMD}{head mounted display}

\newglossaryentry{tacit}{%
  description={Tacit Knowledge is knowledge which is difficult to
    communicate to others via direct methods, such as speaking or
    writing. Examples might include how to play an instrument or playing
    a sport.},
  name={tacit knowledge}
}

\newglossaryentry{psymotor}{%
  description={Psychomotor skills are skills that require extensive coordination between the physical body and mental processes. Examples include playing an instrument or physically manipulating surgical tools during surgery.},
  name={psychomotor skill}
  }

\newglossaryentry{deformablesolid}{%
description={A deformable solid is a volumetric region whose shape reacts to both internal mechanics and external stimuli},
name={deformable solid}
}

\newglossaryentry{hyperelastic}{%
description={A hyperelastic material is a material whose strain energy does not depend on the history of past deformations the material has undergone. While idealized and not found in nature, these materials are often described as rubber-like, as rubber and similar real materials are very hyperelastic in their behavior under low stress loads.},
name={hyperelastic}
}

\newglossaryentry{plasticity}{%
description={A plastic material is a material which incorporates previous deformation history when computing the current deformation strain energy. This gives rise to persistent effects like creasing and shape-holding after bends. Common materials with plasticity effects include most metals, paper, and some plastics.},
name={plasticity}
}

\newacronym[%
description={The Finite Element Method is a type of computational analysis of materials which utilizes a number of discrete regions or elements to describe a larger region of material. By using the relatively simple behaviors of each element, a vastly more complex of the whole material can be derived.}]
{fem}{FEM}{Finite Element Method}